% Options for packages loaded elsewhere
\PassOptionsToPackage{unicode}{hyperref}
\PassOptionsToPackage{hyphens}{url}
\PassOptionsToPackage{dvipsnames,svgnames,x11names}{xcolor}
%
\documentclass[
  letterpaper,
  DIV=11,
  numbers=noendperiod]{scrartcl}

\usepackage{amsmath,amssymb}
\usepackage{lmodern}
\usepackage{iftex}
\ifPDFTeX
  \usepackage[T1]{fontenc}
  \usepackage[utf8]{inputenc}
  \usepackage{textcomp} % provide euro and other symbols
\else % if luatex or xetex
  \usepackage{unicode-math}
  \defaultfontfeatures{Scale=MatchLowercase}
  \defaultfontfeatures[\rmfamily]{Ligatures=TeX,Scale=1}
\fi
% Use upquote if available, for straight quotes in verbatim environments
\IfFileExists{upquote.sty}{\usepackage{upquote}}{}
\IfFileExists{microtype.sty}{% use microtype if available
  \usepackage[]{microtype}
  \UseMicrotypeSet[protrusion]{basicmath} % disable protrusion for tt fonts
}{}
\makeatletter
\@ifundefined{KOMAClassName}{% if non-KOMA class
  \IfFileExists{parskip.sty}{%
    \usepackage{parskip}
  }{% else
    \setlength{\parindent}{0pt}
    \setlength{\parskip}{6pt plus 2pt minus 1pt}}
}{% if KOMA class
  \KOMAoptions{parskip=half}}
\makeatother
\usepackage{xcolor}
\setlength{\emergencystretch}{3em} % prevent overfull lines
\setcounter{secnumdepth}{-\maxdimen} % remove section numbering
% Make \paragraph and \subparagraph free-standing
\ifx\paragraph\undefined\else
  \let\oldparagraph\paragraph
  \renewcommand{\paragraph}[1]{\oldparagraph{#1}\mbox{}}
\fi
\ifx\subparagraph\undefined\else
  \let\oldsubparagraph\subparagraph
  \renewcommand{\subparagraph}[1]{\oldsubparagraph{#1}\mbox{}}
\fi

\usepackage{color}
\usepackage{fancyvrb}
\newcommand{\VerbBar}{|}
\newcommand{\VERB}{\Verb[commandchars=\\\{\}]}
\DefineVerbatimEnvironment{Highlighting}{Verbatim}{commandchars=\\\{\}}
% Add ',fontsize=\small' for more characters per line
\usepackage{framed}
\definecolor{shadecolor}{RGB}{241,243,245}
\newenvironment{Shaded}{\begin{snugshade}}{\end{snugshade}}
\newcommand{\AlertTok}[1]{\textcolor[rgb]{0.68,0.00,0.00}{#1}}
\newcommand{\AnnotationTok}[1]{\textcolor[rgb]{0.37,0.37,0.37}{#1}}
\newcommand{\AttributeTok}[1]{\textcolor[rgb]{0.40,0.45,0.13}{#1}}
\newcommand{\BaseNTok}[1]{\textcolor[rgb]{0.68,0.00,0.00}{#1}}
\newcommand{\BuiltInTok}[1]{\textcolor[rgb]{0.00,0.23,0.31}{#1}}
\newcommand{\CharTok}[1]{\textcolor[rgb]{0.13,0.47,0.30}{#1}}
\newcommand{\CommentTok}[1]{\textcolor[rgb]{0.37,0.37,0.37}{#1}}
\newcommand{\CommentVarTok}[1]{\textcolor[rgb]{0.37,0.37,0.37}{\textit{#1}}}
\newcommand{\ConstantTok}[1]{\textcolor[rgb]{0.56,0.35,0.01}{#1}}
\newcommand{\ControlFlowTok}[1]{\textcolor[rgb]{0.00,0.23,0.31}{#1}}
\newcommand{\DataTypeTok}[1]{\textcolor[rgb]{0.68,0.00,0.00}{#1}}
\newcommand{\DecValTok}[1]{\textcolor[rgb]{0.68,0.00,0.00}{#1}}
\newcommand{\DocumentationTok}[1]{\textcolor[rgb]{0.37,0.37,0.37}{\textit{#1}}}
\newcommand{\ErrorTok}[1]{\textcolor[rgb]{0.68,0.00,0.00}{#1}}
\newcommand{\ExtensionTok}[1]{\textcolor[rgb]{0.00,0.23,0.31}{#1}}
\newcommand{\FloatTok}[1]{\textcolor[rgb]{0.68,0.00,0.00}{#1}}
\newcommand{\FunctionTok}[1]{\textcolor[rgb]{0.28,0.35,0.67}{#1}}
\newcommand{\ImportTok}[1]{\textcolor[rgb]{0.00,0.46,0.62}{#1}}
\newcommand{\InformationTok}[1]{\textcolor[rgb]{0.37,0.37,0.37}{#1}}
\newcommand{\KeywordTok}[1]{\textcolor[rgb]{0.00,0.23,0.31}{#1}}
\newcommand{\NormalTok}[1]{\textcolor[rgb]{0.00,0.23,0.31}{#1}}
\newcommand{\OperatorTok}[1]{\textcolor[rgb]{0.37,0.37,0.37}{#1}}
\newcommand{\OtherTok}[1]{\textcolor[rgb]{0.00,0.23,0.31}{#1}}
\newcommand{\PreprocessorTok}[1]{\textcolor[rgb]{0.68,0.00,0.00}{#1}}
\newcommand{\RegionMarkerTok}[1]{\textcolor[rgb]{0.00,0.23,0.31}{#1}}
\newcommand{\SpecialCharTok}[1]{\textcolor[rgb]{0.37,0.37,0.37}{#1}}
\newcommand{\SpecialStringTok}[1]{\textcolor[rgb]{0.13,0.47,0.30}{#1}}
\newcommand{\StringTok}[1]{\textcolor[rgb]{0.13,0.47,0.30}{#1}}
\newcommand{\VariableTok}[1]{\textcolor[rgb]{0.07,0.07,0.07}{#1}}
\newcommand{\VerbatimStringTok}[1]{\textcolor[rgb]{0.13,0.47,0.30}{#1}}
\newcommand{\WarningTok}[1]{\textcolor[rgb]{0.37,0.37,0.37}{\textit{#1}}}

\providecommand{\tightlist}{%
  \setlength{\itemsep}{0pt}\setlength{\parskip}{0pt}}\usepackage{longtable,booktabs,array}
\usepackage{calc} % for calculating minipage widths
% Correct order of tables after \paragraph or \subparagraph
\usepackage{etoolbox}
\makeatletter
\patchcmd\longtable{\par}{\if@noskipsec\mbox{}\fi\par}{}{}
\makeatother
% Allow footnotes in longtable head/foot
\IfFileExists{footnotehyper.sty}{\usepackage{footnotehyper}}{\usepackage{footnote}}
\makesavenoteenv{longtable}
\usepackage{graphicx}
\makeatletter
\def\maxwidth{\ifdim\Gin@nat@width>\linewidth\linewidth\else\Gin@nat@width\fi}
\def\maxheight{\ifdim\Gin@nat@height>\textheight\textheight\else\Gin@nat@height\fi}
\makeatother
% Scale images if necessary, so that they will not overflow the page
% margins by default, and it is still possible to overwrite the defaults
% using explicit options in \includegraphics[width, height, ...]{}
\setkeys{Gin}{width=\maxwidth,height=\maxheight,keepaspectratio}
% Set default figure placement to htbp
\makeatletter
\def\fps@figure{htbp}
\makeatother

\KOMAoption{captions}{tableheading}
\makeatletter
\makeatother
\makeatletter
\makeatother
\makeatletter
\@ifpackageloaded{caption}{}{\usepackage{caption}}
\AtBeginDocument{%
\ifdefined\contentsname
  \renewcommand*\contentsname{Table of contents}
\else
  \newcommand\contentsname{Table of contents}
\fi
\ifdefined\listfigurename
  \renewcommand*\listfigurename{List of Figures}
\else
  \newcommand\listfigurename{List of Figures}
\fi
\ifdefined\listtablename
  \renewcommand*\listtablename{List of Tables}
\else
  \newcommand\listtablename{List of Tables}
\fi
\ifdefined\figurename
  \renewcommand*\figurename{Figure}
\else
  \newcommand\figurename{Figure}
\fi
\ifdefined\tablename
  \renewcommand*\tablename{Table}
\else
  \newcommand\tablename{Table}
\fi
}
\@ifpackageloaded{float}{}{\usepackage{float}}
\floatstyle{ruled}
\@ifundefined{c@chapter}{\newfloat{codelisting}{h}{lop}}{\newfloat{codelisting}{h}{lop}[chapter]}
\floatname{codelisting}{Listing}
\newcommand*\listoflistings{\listof{codelisting}{List of Listings}}
\makeatother
\makeatletter
\@ifpackageloaded{caption}{}{\usepackage{caption}}
\@ifpackageloaded{subcaption}{}{\usepackage{subcaption}}
\makeatother
\makeatletter
\@ifpackageloaded{tcolorbox}{}{\usepackage[many]{tcolorbox}}
\makeatother
\makeatletter
\@ifundefined{shadecolor}{\definecolor{shadecolor}{rgb}{.97, .97, .97}}
\makeatother
\makeatletter
\makeatother
\ifLuaTeX
  \usepackage{selnolig}  % disable illegal ligatures
\fi
\IfFileExists{bookmark.sty}{\usepackage{bookmark}}{\usepackage{hyperref}}
\IfFileExists{xurl.sty}{\usepackage{xurl}}{} % add URL line breaks if available
\urlstyle{same} % disable monospaced font for URLs
\hypersetup{
  pdftitle={STA 532 Final},
  pdfauthor={Will Tirone},
  colorlinks=true,
  linkcolor={blue},
  filecolor={Maroon},
  citecolor={Blue},
  urlcolor={Blue},
  pdfcreator={LaTeX via pandoc}}

\title{STA 532 Final}
\author{Will Tirone}
\date{5/1/23}

\begin{document}
\maketitle
\ifdefined\Shaded\renewenvironment{Shaded}{\begin{tcolorbox}[boxrule=0pt, interior hidden, sharp corners, frame hidden, enhanced, breakable, borderline west={3pt}{0pt}{shadecolor}]}{\end{tcolorbox}}\fi

\begin{verbatim}
    j Y_j
1   1 118
2   2  74
3   3  44
4   4  24
5   5  29
6   6  22
7   7  20
8   8  19
9   9  20
10 10  15
11 11  12
12 12  14
13 13   6
14 14  12
15 15   6
16 16   9
17 17   9
18 18   6
19 19  10
20 20  10
21 21  11
22 22   5
23 23   3
24 24   3
\end{verbatim}

\hypertarget{section}{%
\section{5)}\label{section}}

\begin{Shaded}
\begin{Highlighting}[]
\CommentTok{\# j }
\NormalTok{n }\OtherTok{=} \DecValTok{501}

\CommentTok{\# define MLE, for c(k,u)}
\NormalTok{mle }\OtherTok{=} \FunctionTok{c}\NormalTok{(}\FloatTok{0.491}\NormalTok{, }\FloatTok{4.48}\NormalTok{)}

\CommentTok{\# define normalized Hessian from problem}
\NormalTok{normalized\_H }\OtherTok{=} \FunctionTok{matrix}\NormalTok{(}\FunctionTok{c}\NormalTok{(}\FloatTok{305.82}\NormalTok{, }\SpecialCharTok{{-}}\FloatTok{48.71}\NormalTok{, }\SpecialCharTok{{-}}\FloatTok{48.71}\NormalTok{, }\FloatTok{12.86}\NormalTok{), }\DecValTok{2}\NormalTok{,}\DecValTok{2}\NormalTok{,}\AttributeTok{byrow=}\NormalTok{T)}\SpecialCharTok{/}\NormalTok{n}

\CommentTok{\# define pi which we will differentiate}
\NormalTok{pi\_func }\OtherTok{=} \FunctionTok{expression}\NormalTok{((k }\SpecialCharTok{/}\NormalTok{ (u }\SpecialCharTok{+}\NormalTok{ k))}\SpecialCharTok{\^{}}\NormalTok{k)}

\NormalTok{pi\_deriv\_k }\OtherTok{=} \FunctionTok{D}\NormalTok{(pi\_func, }\StringTok{\textquotesingle{}k\textquotesingle{}}\NormalTok{)}
\NormalTok{pi\_deriv\_u }\OtherTok{=} \FunctionTok{D}\NormalTok{(pi\_func, }\StringTok{\textquotesingle{}u\textquotesingle{}}\NormalTok{)}

\NormalTok{k }\OtherTok{=}\NormalTok{ mle[}\DecValTok{1}\NormalTok{]}
\NormalTok{u }\OtherTok{=}\NormalTok{ mle[}\DecValTok{2}\NormalTok{]}

\NormalTok{deriv\_vector }\OtherTok{=} \FunctionTok{matrix}\NormalTok{(}\FunctionTok{c}\NormalTok{(}\FunctionTok{eval}\NormalTok{(pi\_deriv\_k), }\FunctionTok{eval}\NormalTok{(pi\_deriv\_u)))}

\NormalTok{mle\_asymptotic\_variance }\OtherTok{=} \FunctionTok{t}\NormalTok{(deriv\_vector) }\SpecialCharTok{\%*\%}\NormalTok{ normalized\_H }\SpecialCharTok{\%*\%}\NormalTok{ deriv\_vector}
\NormalTok{pi\_hat }\OtherTok{=} \FunctionTok{eval}\NormalTok{(pi\_func)}

\CommentTok{\# print results }
\FunctionTok{cat}\NormalTok{(}\StringTok{"Asymptotic variance of MLE :"}\NormalTok{, mle\_asymptotic\_variance)}
\end{Highlighting}
\end{Shaded}

\begin{verbatim}
Asymptotic variance of MLE : 0.1228559
\end{verbatim}

\begin{Shaded}
\begin{Highlighting}[]
\FunctionTok{cat}\NormalTok{(}\StringTok{"}\SpecialCharTok{\textbackslash{}n}\StringTok{"}\NormalTok{)}
\end{Highlighting}
\end{Shaded}

\begin{Shaded}
\begin{Highlighting}[]
\FunctionTok{cat}\NormalTok{(}\StringTok{"Asymptotic SD of MLE :"}\NormalTok{, }\FunctionTok{sqrt}\NormalTok{(mle\_asymptotic\_variance))}
\end{Highlighting}
\end{Shaded}

\begin{verbatim}
Asymptotic SD of MLE : 0.3505081
\end{verbatim}

\begin{Shaded}
\begin{Highlighting}[]
\FunctionTok{cat}\NormalTok{(}\StringTok{"}\SpecialCharTok{\textbackslash{}n}\StringTok{"}\NormalTok{)}
\end{Highlighting}
\end{Shaded}

\begin{Shaded}
\begin{Highlighting}[]
\FunctionTok{cat}\NormalTok{(}\StringTok{"pi evaluated at MLE : "}\NormalTok{, pi\_hat)}
\end{Highlighting}
\end{Shaded}

\begin{verbatim}
pi evaluated at MLE :  0.3208981
\end{verbatim}

Numerically finding the confidence interval:

\begin{Shaded}
\begin{Highlighting}[]
\NormalTok{moe }\OtherTok{=} \FloatTok{2.24} \SpecialCharTok{*}\NormalTok{ (}\FloatTok{1.601}\SpecialCharTok{/}\FunctionTok{sqrt}\NormalTok{(n))}
\NormalTok{pi\_hat }\SpecialCharTok{{-}}\NormalTok{ moe  }
\end{Highlighting}
\end{Shaded}

\begin{verbatim}
[1] 0.1606767
\end{verbatim}

\begin{Shaded}
\begin{Highlighting}[]
\NormalTok{pi\_hat }\SpecialCharTok{+}\NormalTok{ moe}
\end{Highlighting}
\end{Shaded}

\begin{verbatim}
[1] 0.4811195
\end{verbatim}

\begin{Shaded}
\begin{Highlighting}[]
\CommentTok{\# checking 4.5}

\NormalTok{f }\OtherTok{=} \FunctionTok{expression}\NormalTok{(}\FunctionTok{exp}\NormalTok{(lambda }\SpecialCharTok{+}\NormalTok{ (kappa}\SpecialCharTok{/}\DecValTok{2}\NormalTok{)))}
\NormalTok{l }\OtherTok{=} \FunctionTok{D}\NormalTok{(f, }\StringTok{\textquotesingle{}lambda\textquotesingle{}}\NormalTok{)}
\NormalTok{kap }\OtherTok{=} \FunctionTok{D}\NormalTok{(f, }\StringTok{\textquotesingle{}kappa\textquotesingle{}}\NormalTok{)}

\NormalTok{lambda }\OtherTok{=} \FloatTok{4.93}
\NormalTok{kappa }\OtherTok{=} \FloatTok{0.07}

\FunctionTok{eval}\NormalTok{(f)}
\end{Highlighting}
\end{Shaded}

\begin{verbatim}
[1] 143.3086
\end{verbatim}

\begin{Shaded}
\begin{Highlighting}[]
\NormalTok{m }\OtherTok{=} \FunctionTok{matrix}\NormalTok{(}\FunctionTok{c}\NormalTok{(}\FunctionTok{eval}\NormalTok{(l), }\FunctionTok{eval}\NormalTok{(kap)))}
\NormalTok{I }\OtherTok{=} \FunctionTok{matrix}\NormalTok{(}\FunctionTok{c}\NormalTok{(}\FloatTok{0.07}\NormalTok{, }\DecValTok{0}\NormalTok{, }\DecValTok{0}\NormalTok{, }\DecValTok{2} \SpecialCharTok{*}\NormalTok{ (}\FloatTok{0.07}\NormalTok{)}\SpecialCharTok{\^{}}\DecValTok{2}\NormalTok{),}\DecValTok{2}\NormalTok{,}\DecValTok{2}\NormalTok{,}\AttributeTok{byrow=}\NormalTok{T)}
\FunctionTok{t}\NormalTok{(m) }\SpecialCharTok{\%*\%}\NormalTok{ I }\SpecialCharTok{\%*\%}\NormalTok{ m}
\end{Highlighting}
\end{Shaded}

\begin{verbatim}
        [,1]
[1,] 1487.93
\end{verbatim}

\begin{Shaded}
\begin{Highlighting}[]
\FunctionTok{exp}\NormalTok{(}\DecValTok{2} \SpecialCharTok{*} \FloatTok{4.93} \SpecialCharTok{+} \FloatTok{0.07}\NormalTok{) }\SpecialCharTok{*}\NormalTok{ (}\FloatTok{0.07} \SpecialCharTok{+}\NormalTok{ (}\FloatTok{0.5} \SpecialCharTok{*} \FloatTok{0.07}\NormalTok{)}\SpecialCharTok{\^{}}\DecValTok{2}\NormalTok{)}
\end{Highlighting}
\end{Shaded}

\begin{verbatim}
[1] 1462.772
\end{verbatim}

\hypertarget{trying-to-show-convergence}{%
\subsection{7) Trying to show
convergence}\label{trying-to-show-convergence}}

\begin{Shaded}
\begin{Highlighting}[]
\NormalTok{eps\_range }\OtherTok{=} \FunctionTok{c}\NormalTok{(}\FloatTok{0.1}\NormalTok{,}\FloatTok{0.01}\NormalTok{,}\FloatTok{0.001}\NormalTok{,}\FloatTok{0.0001}\NormalTok{)}
\NormalTok{rep }\OtherTok{=} \DecValTok{1000} \CommentTok{\# number of MC repetitions}
\NormalTok{pi\_hat }\OtherTok{=} \FloatTok{0.321} \CommentTok{\# MLE estimate}

\NormalTok{S }\OtherTok{=} \DecValTok{100000000000}
\NormalTok{mc\_values }\OtherTok{=} \FunctionTok{c}\NormalTok{()}
\ControlFlowTok{for}\NormalTok{ (epsilon }\ControlFlowTok{in}\NormalTok{ eps\_range)\{}
\NormalTok{  probability\_eval }\OtherTok{=} \FunctionTok{c}\NormalTok{()}
  \ControlFlowTok{for}\NormalTok{ (i }\ControlFlowTok{in} \DecValTok{1}\SpecialCharTok{:}\NormalTok{rep)\{}
    \CommentTok{\# draw}
\NormalTok{    M }\OtherTok{=} \FunctionTok{rbinom}\NormalTok{(}\DecValTok{1}\NormalTok{,S,(}\DecValTok{1}\SpecialCharTok{{-}}\NormalTok{pi\_hat))}
\NormalTok{    S\_hat }\OtherTok{=}\NormalTok{ M }\SpecialCharTok{/}\NormalTok{ (}\DecValTok{1}\SpecialCharTok{{-}}\NormalTok{pi\_hat)}
    
    \CommentTok{\# evaluate probability and store }
\NormalTok{    val }\OtherTok{=} \FunctionTok{abs}\NormalTok{(S\_hat }\SpecialCharTok{/}\NormalTok{ S }\SpecialCharTok{{-}} \DecValTok{1}\NormalTok{) }\SpecialCharTok{\textgreater{}}\NormalTok{ epsilon}
\NormalTok{    probability\_eval }\OtherTok{=} \FunctionTok{c}\NormalTok{(probability\_eval, val)}
\NormalTok{  \}}
\NormalTok{  mc\_est }\OtherTok{=} \FunctionTok{mean}\NormalTok{(probability\_eval)}
\NormalTok{  mc\_values }\OtherTok{=} \FunctionTok{c}\NormalTok{(mc\_values, mc\_est)}
\NormalTok{\}}

\NormalTok{mc\_values}
\end{Highlighting}
\end{Shaded}

\begin{verbatim}
[1] 0 0 0 0
\end{verbatim}

\begin{aligned}
\text{coverage}(\gamma; \pi) &= P(\hat{\pi} - 2.24\frac{\hat{\sigma}}{\sqrt{n}} < \pi < \hat{\pi} + 2.24\frac{\hat{\sigma}}{\sqrt{n}} | \pi) \\
&= P(-2.24 < \frac{\hat{\pi} - \pi}{\hat{\sigma} / \sqrt{n}}  < 2.24 | \pi) \\ 
&= \Phi(2.24) - \Phi(-2.24) \approx 97.5\%
\end{aligned}

\begin{Shaded}
\begin{Highlighting}[]
\FunctionTok{sqrt}\NormalTok{(}\FloatTok{0.122}\NormalTok{)}
\end{Highlighting}
\end{Shaded}

\begin{verbatim}
[1] 0.349285
\end{verbatim}



\end{document}
